\documentclass[11pt,a4paper]{article}
\usepackage[utf8]{inputenc}
\usepackage{geometry}
\geometry{margin=1in}
\usepackage{graphicx}
\usepackage{amsmath}
\usepackage{hyperref}

\title{Procedural Generation of Urban Environments with Minimal Urban Planning Rules}
\author{Dmitry Novozhilov}
\date{January 2026}

\begin{document}

\maketitle

\begin{abstract}
Procedural generation is a powerful technique for synthesising complex content from
compact rules and random seeds.  In the context of urban environments this
approach allows one to create vast, unique cityscapes on demand, useful for
games, simulations and urban planning studies[17†L19-L27].  We present a
modular prototype for procedural city generation that transforms high‑level
parameters—population, numbers of hospitals and schools, a primary transport
mode and a random seed—into a complete three‑dimensional model of a city.
Our generator adheres to a handful of basic urban planning guidelines, such
as minimum green space per inhabitant[26†L7-L10] and the distribution of
educational and healthcare facilities[25†L825-L834].  The implementation
employs C++ for performance and a Python wrapper for ease of use, and
includes integration tests verifying determinism and rule compliance.  This
paper summarises the design, algorithm and implementation of the generator.
\end{abstract}

\section{Introduction}
Cities arise from a complex interplay of geography, economics and societal
forces.  Modelling such complexity procedurally requires simplifying
assumptions and rules that nevertheless produce believable results.  Prior
work on procedural city generation has explored grid‑based and agent
approaches[17†L17-L25], multi‑zone models inspired by the Burgess concentric
zone theory[18†L67-L75], and noise‑driven growth akin to terrain
generation in Minecraft[21†L92-L100].  Our goal is to design a generator
that, while simplified, produces cities that obey some well‑known urban
planning principles and remain reproducible under a fixed seed.

\section{Input Parameters}
The generator accepts several high‑level parameters:

\begin{itemize}
  \item \textbf{Population.}  The number of inhabitants in the city influences
        the spatial extent of development.  A larger population implies a
        larger buildable area and more infrastructure.  The gross density is
        assumed implicitly, but can be tuned via the grid size and radius.
  \item \textbf{Number of social facilities.}  Users specify how many
        hospitals and primary schools should exist.  These facilities are
        distributed across residential and commercial zones to ensure
        coverage; in practice one hospital per about 100~000–120~000 people is
        recommended[14†L318-L326] and one primary school per 6~000–10~000
        residents[25†L825-L834].
  \item \textbf{Primary transport mode.}  The dominant mode of transport
        (car, transit or walking) influences the road network layout.  In this
        prototype we generate a simple cross and two ring roads, but the
        parameter is provided for future extensions (e.g. generating a
        transit line when the mode is ``transit'').
  \item \textbf{Random seed.}  Procedural content should be repeatable; given
        identical parameters and a seed, the same city must be produced.  Our
        generator uses a 32‑bit seed to initialise all random processes.
  \item \textbf{Grid size and radius fraction.}  The city is discretised into
        a square grid of \texttt{gridSize}~$\times$~\texttt{gridSize} cells.
        A radial boundary defined by \texttt{radiusFraction} of half the
        grid extent delimits the developed area; outside cells remain
        undeveloped (zone ``None'').
\end{itemize}

\section{Urban Planning Rules}
While a complete simulation of urbanism is beyond scope, we encode several
basic planning rules drawn from the literature:

\paragraph{Green space per inhabitant.}  World Health Organization guidelines
recommend at least 8~m$^2$ of accessible green space per inhabitant in
cities[26†L7-L10].  Our generator calculates the total area of green cells
given an assumed cell area (100~m~$\times$~100~m) and the population; if the
initial zoning yields too few green cells, random residential or industrial
lots are converted to parks until the target is met.

\paragraph{Education and health.}  Access to schools and healthcare should be
ubiquitous.  Primary schools are typically located within 500–1000~m of
residences and serve roughly 6~000–10~000 people[25†L825-L834].  Hospitals
should serve about 100~000–120~000 people[14†L318-L326].  We simply place
the user‑specified number of facilities at random within residential and
commercial zones; more sophisticated placement (e.g. ensuring maximum
distance constraints) is a topic for future work.

\section{Algorithm Overview}
The generation pipeline follows the steps below:

\begin{enumerate}
  \item \textbf{Initial zoning.}  Each grid cell inside the radial boundary
        receives a noise value from a multi‑octave hash‑based noise function.
        The value is mapped to one of four zones: residential, commercial,
        industrial or green, based on predefined thresholds.  Cells outside
        the radius remain undeveloped.  Building heights are sampled
        stochastically from zone‑specific ranges (e.g. 2–6 floors for
        residential, 5–20 for commercial).
  \item \textbf{Green space enforcement.}  The number of green cells is
        compared to the required amount derived from the population and cell
        area.  If insufficient, random residential or industrial cells are
        converted to parks until the requirement is satisfied.
  \item \textbf{Facility placement.}  Hospitals and schools are placed by
        sampling random eligible lots (residential or commercial) without
        replacement until the requested counts are met.  Facilities are
        flagged so tests can verify their counts.
  \item \textbf{Road network.}  The road system comprises a cross of
        orthogonal boulevards through the city centre and two concentric
        ring roads at 50\% and 90\% of the developed radius.  Each road is
        represented as a set of line segments; more complex networks could
        be generated following the radial or grid patterns discussed in
        prior work[17†L17-L25].
  \item \textbf{Output.}  The final city is serialised into two formats: a
        Wavefront OBJ mesh where each non‑green lot becomes a cuboid, and a
        JSON summary containing high‑level counts for testing and analysis.
\end{enumerate}

\section{Implementation}
The core of the generator is implemented in modern C++ and organised into
three main classes:

\begin{itemize}
  \item \texttt{Config} – holds all user‑supplied parameters.
  \item \texttt{City} – stores the resulting grid of buildings, the list of
        facilities and the road segments.  Methods allow for OBJ and JSON
        serialisation.
  \item \texttt{CityGenerator} – exposes a static \texttt{generate} method
        that performs the pipeline described above using the configuration.
\end{itemize}

Noise is generated using a simple hash of grid coordinates and the seed.
Although less smooth than Perlin or Simplex noise, this method has no
external dependencies and suffices for the prototype.  A fractal sum of
four octaves introduces multi‑scale variation.  The generator uses the
\texttt{std::mt19937} engine for deterministic random choices, ensuring
repeatability.  A command‑line interface built in \texttt{main.cpp} parses
arguments and writes the outputs to a specified directory, while a
Python wrapper simplifies invocation from scripts.

\section{Testing}
Integration tests written in Python validate key properties of the
implementation.  The tests compile the C++ code, run the generator with
various configurations and inspect the JSON summaries.  They verify that
identical seeds yield identical summaries (determinism), that the counts of
hospitals and schools match the configuration, and that the green space
allocated meets or exceeds the recommended minimum per capita[26†L7-L10].
Future tests could enforce maximum walking distances to facilities and
compare generated road networks against reference patterns.

\section{Conclusion and Future Work}
We have presented a modular procedural city generator that produces
reproducible urban models from a handful of parameters.  By embedding
simple urban planning rules—minimum green space, basic facility provision
and a rudimentary road network—the generated cities avoid some of the
pathologies of purely random constructions.  Nevertheless, there is
tremendous scope for refinement: incorporating more realistic noise,
simulating growth dynamics[22†L23-L39], enforcing service access
constraints, and adding variety in building geometry.  We hope this
prototype serves as a foundation for further experimentation and as a
teaching aid for procedural content generation.

\begin{thebibliography}{9}

\bibitem{Sahinoglu2022}
Sahinoglu, B. Y., \& Celikcan, U. (2022). A grid‑based multi‑zone Burgess
approach for fast procedural city generation from scratch. \emph{Mugla
Journal of Science and Technology}, 8(1), 8–18.[17†L17-L25][18†L67-L75]

\bibitem{WHO2024}
World Health Organization.  (2024).  Guidelines for healthcare facilities.
\emph{BMJ Open}, 14, e000000.[14†L318-L326]

\bibitem{MDPI2023}
MDPI Sustainability Journal.  (2023).  Role of urban planning standards in
improving lifestyle in a sustainable system.  Sections on infrastructure
planning standards and green space requirements.[25†L825-L834][26†L7-L10]

\bibitem{MinecraftTerrain}
Cybrancee Blog.  (2025).  How Minecraft terrain generation works.  Retrieved
from the author's website.[21†L92-L100]

\end{thebibliography}

\end{document}